\documentclass[ngerman,a4paper,10pt,headsepline,twoside,twocolumn]{article}

%----------------- PDF CONFIG ----------------- %
\pdfinfo{    
     /Title (PDF-Titel) 
     /Subject   (PDF-Thema)    
     /Author  (Vorname Nachname) 
     /Keywords   (Stichwort1,Stichwort2)      
} 

\title{SPH}
\author{Can Kedik}
\date{\today}



%----------------- PAKETE INKLUDIEREN ----------------- %

\usepackage{geometry} % Packet für Seitenrandabständex und Einstellung für Seitenränder
\usepackage[ngerman]{babel} % deutsche Silbentrennung
\usepackage[verbose]{placeins}

\usepackage{booktabs} %entzerrt die Tabellenzeilen und bietet verschieden dicke Unterteilungslinien
\usepackage{longtable}% Tabellen können sich nicht über mehrere Seiten 
\usepackage{graphicx} % kann LaTeX Grafiken einbinden

\usepackage{setspace}
\usepackage{mathabx} 
\usepackage[applemac]{inputenc} % Umlaute unter Mac werden automatisch gesetzt
\usepackage[T1]{fontenc} % Zeichenencoding
\usepackage{lmodern} % typographische Qualität 
\frenchspacing % Schaltet den zusätzlichen Zwischenraum ab
\usepackage{fix-cm}
\usepackage{varioref}
\usepackage{hyperref} % verwandelt alle Kapitelüberschriften, Verweise aufs Literaturverzeichnis und andere Querverweise in PDF-Hyperlinks
\usepackage{cleveref}
\usepackage{color}
\usepackage{url}
\usepackage[babel, german=quotes]{csquotes}
\usepackage{array}
\usepackage{siunitx}
\usepackage[backend=biber,style=alphabetic]{biblatex}
\bibliography{literatur}


\usepackage[nottoc]{tocbibind}



% für Listings
\usepackage{listings}
\lstset{numbers=left, numberstyle=\tiny, numbersep=5pt, stepnumber=4, keywordstyle=\color{black}\bfseries\itshape, stringstyle=\ttfamily,showstringspaces=false,basicstyle=\footnotesize,captionpos=b}
\lstset{language=java}



%----------------- FARBEN DEFINIEREN ----------------- %
\definecolor{gray}{gray}{0.95} % Listingsbackground

%----------------- LAYOUT SETZEN ----------------- %
\geometry{left=2.5cm, right=2cm, top=2.5cm, bottom=2cm}
\linespread {1.25}\selectfont %1.25 da er von Haus aus 1.2 ist und 1,25 * 1,2 = 1,5 isch



%-------##-------##-------##------- ANFANG INHALT -------##-------##-------##-------%
\begin{document}



\pagenumbering{roman} % Seitennummer

%----------------- DECKBLATT -----------------%

%----------------- ABSTRACT -----------------%
%%----------------- KONFIGURATION ----------------- %
%\pagestyle{empty} % enthalten keinerlei Kopf oder Fuß
\pagenumbering{arabic}


\section*{Abstract} % (fold)
\label{cha:abtract}

Der folgende Artikel setzt sich mit der Umsetzung der %
der sogenannten Smoothed Particle Hydrodynamics Methode %
zur Simulation von Fl"ussigkeiten ausseinander. %
Die Umsetzung der SPH wurde mit der Spiele Engine Unity %
gemacht die programmierung erfolgte dann in C\#. %
In diesem Artikel werden die Mathematischen Grundlagen %
der SPH erkl"art. In der Diskussion wird betrachtet ob %
die SPH sich auch f"ur andere Fl"ussigkeiten eignet.

\section{Einleitung}
\label{sec:einleitung}

In diesem Abschnitt wird die Motivation erkl"art. %

\section{Verwandte Arbeiten}
\label{sec:arbeiten}


\section{Smoothed Particle Hydrodynamics}
\label{sec:sph}
Hier sind die Formeln der SPH und was sie machen erl"autert % 

\section{Umsetzung in Unity}
\label{sec:umsetzung}

\section{Diskussion}
\label{sec:diskussion}

\section{Fazit}
\label{sec:fazit}

 
 
%----------------- VERZEICHNISSE -----------------%

% ----- Abbildungen ----- %
%\addcontentsline{toc}{section}{Abbildungsverzeichnis} % falls in Inhalsverzeichnis

% ----- Tabellen----- %
% \addcontentsline{toc}{section}{Tabellenverzeichnis}  % falls in Inhalsverzeichnis
% \fancyhead[L]{Abbildungsverzeichnis / Abkürzungsverzeichnis} %Kopfzeile links

% ----- Listings ----- %
% Listingverzeichnis soll im Inhaltsverzeichnis auftauchen
% \addcontentsline{toc}{section}{Listingverzeichnis}
% \fancyhead[L]{Abbildungs- / Tabellen- / Listingverzeichnis} %Kopfzeile links
%\renewcommand{\lstlistlistingname}{Listingverzeichnis}
%\lstlistoflistings

\pagestyle{headings} % zurueck setzen von roemische seitenanzahl

%----------------- KONFIGURATION ----------------- %
%\pagestyle{empty} % enthalten keinerlei Kopf oder Fuß
\pagenumbering{arabic}


\section*{Abstract} % (fold)
\label{cha:abtract}

Der folgende Artikel setzt sich mit der Umsetzung der %
der sogenannten Smoothed Particle Hydrodynamics Methode %
zur Simulation von Fl"ussigkeiten ausseinander. %
Die Umsetzung der SPH wurde mit der Spiele Engine Unity %
gemacht die programmierung erfolgte dann in C\#. %
In diesem Artikel werden die Mathematischen Grundlagen %
der SPH erkl"art. In der Diskussion wird betrachtet ob %
die SPH sich auch f"ur andere Fl"ussigkeiten eignet.

\section{Einleitung}
\label{sec:einleitung}

In diesem Abschnitt wird die Motivation erkl"art. %

\section{Verwandte Arbeiten}
\label{sec:arbeiten}


\section{Smoothed Particle Hydrodynamics}
\label{sec:sph}
Hier sind die Formeln der SPH und was sie machen erl"autert % 

\section{Umsetzung in Unity}
\label{sec:umsetzung}

\section{Diskussion}
\label{sec:diskussion}

\section{Fazit}
\label{sec:fazit}



%----------------- KAPITEL : EINLEITUNG  ----------------- %
\section{Einleitung}
\label{sec:einleitung}

Die realistische Simulation von Fl"u"sigkeiten stellte in der %
Computer Graphik einen spannenden Teil dar. 


%----------------- KAPITEL : FRAMEWORK  ----------------- %
\section{Smoothed Particle Hydrodynamics}
\label{sec:sph}

Das Smoothed Particle Hydrodynamics kurz SPH stellt eine numerische Methode %
zum berechnen von Hydrodynamischen Gleichungen dar. Dabei gilt sie als %
besonders einfach zu implementierende und robuste Methode. %
Die SPH Methode ist zur Simulation von vielen Physikalischen Eigenschaften %
geeignet so kann sie auch zur Simulation von Astrophysikalischen Eigenschaften %
verwendet werden. % 
Jedoch ist die SPH nicht nur auf die Simulation von Wasser beschr"ankt sondern %
ist es auch m"oglich andere Fl"u"sigkeiten sowie auch Rauch darzustellen. %
Dabei ist jedoch anzumerken das die SPH eine rein empirische Methode ist %
sie wird angewendet weil sie f"ur ihren Zweck funktionable ist, %
nicht jedoch weil sie eine Realit"atsgetreue Abbildung ist. %


Im folgenden Abschnitt m"ochte ich die n"otigen Mathematischen Funktionen erl"autern %
die daf"ur n"otig sind die SPH zu implementieren und die Grundlagen darstellen. %



%----------------- KAPITEL : TESTAUFBAU  ----------------- %
%----------------- KAPITEL : VERSUCHE  ----------------- %
%----------------- KAPITEL : VERGLEICH------------ %	

%----------------- KAPITEL : FAZIT  ----------------- %	
%\chapter{Fazit}
%----------------- KAPITEL : LITERATUR  ----------------- %	
\printbibliography
%----------------- KAPITEL : ANHANG  ----------------- %	


\end{document}
